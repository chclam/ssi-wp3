\documentclass{article}

\usepackage{graphicx}

\title{SSI WP3: Project report}

\author{Tim De Jong}

\begin{document}

\maketitle


\section{Introduction}

Process:
- Data preparation:
    - LIDL revenue data already contains receipt texts, EAN name = receipt text
    - Plus revenue data does not contain receipt texts, the receipt texts are in another file and need to became
    coupled on an internal identifier called rep id. Plus revenue data is available for a larger time period than the
    Plus receipt data. What do we do with the data before the receipt texts are available?

- Data analysis
- Train ML classifier
- Evaluate ML classifier:
    - After training
    - On new data (for instance different supermarket)
    - Over time (train on one year, evaluate on next year)




Train a ML classifier to classify receipt texts to COICOP labels:
- What are common ways to classify short texts?
- Several levels of COICOP
    - One classifier for deepest COICOP level?
    - One classifier for each COICOP level? If so, how to combine these classifiers?
    - Can we use sales data as a prior for the classifier? For instance, number of products sold, 
    or the revenue of a product?
    - Which COICOP level is sufficient for our purposes? i.e. the SSI/HBS app and survey?
- Receipt text for Plus and LIDL supermarkets
- Encode receipt text as a vector, several methods: 
    - Several encodings have different qualities, evaluate quality of these encodings
    - Which encoding/embedding works the best?
    - How do vector distances differ between COICOP categories? How? Which base-model?
- Do receipt text change COICOP label and if so how often?
- How often do unique identifiers such as EAN codes change? On what COICOP level do they change?
- What about the supermarkets for which we do not have data? 
- What about other stores for which we do not have any labeled data?
- Does adding the product labels from the ean\_name column improve the classifier?
- How much data do we need to train a classifier?
- Which metric to use to evaluate the classifier?
    
Can we find new products easily? Outlier/Novelty detection methods:    
    - Cluster on vectors of receipt text:
        - Check labels of text with cluster labels and see if some receipt texts should be part of another cluster based on the vector
        this gives an idea of the quality of the COICOP classification but can also give an idea about the quality of the embedding
        - Paul Keuren: use OPTICS clustering algorithm
- Train on one supermarket, evaluate on other supermarket
- Which product categories are most difficult to classify?
- Which product caterogies are classified best/worst?
- How do supermarkets differ in their product categories?

String distances:
- How do string distances differ between COICOP categories?

Manual labeling by participants:
- How do we show participants a list of relevant COICOP categories?
    - Which distance measure is best to use?
        - String distance (Levenshtein, Jaro-Winkler, etc.)
        - Vector distance (cosine, euclidean, etc.) in a vector space of receipt texts, and then which embedding?
- Which level of COICOP category?
- Can we evaluate the quality of the manual labels? By comparing random selection of top k-labels with a 
bootstrap method?

From previous work it became clear that ML classifier performance degrades over time
- Evaluate degradation of performance over time:
    - Train on one year, evaluate on next year




\end{document}